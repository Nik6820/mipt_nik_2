\documentclass[a4paper]{article}
\usepackage[warn]{mathtext}
\usepackage{graphicx} % Required for inserting images
\usepackage{wrapfig}
\usepackage{enumitem}
\usepackage[center]{caption}
\usepackage[english, russian]{babel}
\usepackage{amsmath}
\usepackage{geometry}
\setcounter{totalnumber}{5}

\geometry{verbose, a4paper, tmargin=2cm, bmargin=2cm, lmargin=2.0cm, rmargin=2.0cm}

\begin{document}
	
	\begin{titlepage}
		\centering
		{\scshape\Large МОСКОВСКИЙ ФИЗИКО-ТЕХНИЧЕСКИЙ ИНСТИТУТ \\
			(НАЦИОНАЛЬНЫЙ ИССЛЕДОВАТЕЛЬСКИЙ УНИВЕРСИТЕТ)\\ % название ВУЗа большим шрифтом
			Физтех-школа аэрокосмических технологий}
		
		\vspace{4cm} % отступ 4 см по вертикали
		{\LARGE Отчет о выполнении лабораторной работы 1.3.3}
		\vspace{1cm} % ещё 1 см
		
		{\huge\bf Измерение вязкости воздуха по течению в тонких трубках}
		
		\vspace{1cm} % ещё 1 см
		\vfill % заполнение по вертикали пустотой (LaTeX сам решит, сколько здесь отступить)
		
		\begin{flushright} % Этот текст будет с правого края страницы
			{\LARGE Губарев Никита, Б03-502}
		\end{flushright}
		\vfill 
		\today % Дата сборки документа
	\end{titlepage}
	\tableofcontents 
\section{Аннотация}
	\textbf{Цель работы:} Экспериментально исследовать свойства течения газов по тонким трубкам при различных числах Рейнольдса; выявить область применимости закона Пуазейля и с его помощью определить коэффициент вязкости воздуха.
	
	\textbf{Оборудование:} Система подачи воздуха (компрессор, проводящие трубки); газовый счетчик барабанного типа; спиртовой микроманометр с регулируемым наклоном; набор трубок различного диаметра с выходами для подсоединения микроманометра; секундомер.
	
	\section{Теоретические сведения}
	
	Движение жидкости или газа вызывается перепадом внешнего давления на концах $\Delta P$, чему препятствуют силы вязкого («внутреннего») трения, действующие между соседними слоями жидкости, со стороны стенок трубы.
	
	Сила вязкого трения описывается законом Ньютона: касательное напряжение между слоями пропорционально перепаду скорости течения в направлении, поперечном к потоку. В частности, если жидкость течёт вдоль оси $x$, а скорость течения $v_x(y)$ зависит от координаты $y$, в каждом слое возникает направленное по $x$ касательное напряжение
	
	\begin{equation}
		\tau_{xy} = -\eta \frac{\partial v_x}{\partial y}. \tag{1}
	\end{equation}
	
	Величина $\eta$ — коэффициент динамической вязкости среды.
	
	Объёмный расход $Q$ — объём жидкости, протекающий через сечение трубы в единицу времени. Величина $Q$ зависит от перепада давления $\Delta P$, а также от свойств газа (плотности $\rho$ и вязкости $\eta$) и от геометрических размеров (радиуса трубы $R$ и её длины $L$).
	
	При ламинарном течении поле скоростей $u(r)$ образует набор непрерывных линий тока, а слои жидкости не перемешиваются между собой. Турбулентное течение характеризуется образованием вихрей и активным перемешиванием слоев.
	
	Характер течения определяется безразмерным параметром задачи — числом Рейнольдса:
	
	\begin{equation}
		\text{Re} = \frac{\rho u a}{\eta}, \tag{2}
	\end{equation}
	
	Где $\rho$ — плотность среды, $u$ — характерная скорость потока, $\eta$ — коэффициент вязкости среды, $a$ — характерный размер системы (размер, на котором существенно меняется скорость течения). Это число имеет смысл отношения кинетической энергии движения элемента объёма жидкости к потерям энергии из-за трения в нём: $Re \approx K/A_тр$. При достаточно малых Re в потоке доминируют вязкие силы трения и течение, как правило, является ламинарным. С ростом числа Рейнольдса может быть достигнуто его критическое значение Reкр, при котором характер течения сменяется с ламинарного на турбулентный.
	
	\subsection{Течение Пуазейля}
	
	При малых числах Рейнольдса течение в прямой трубе с гладкими стенками имеет ламинарный характер. Для установившегося ламинарного течения несжимаемой жидкости в круглой трубе поле скоростей имеет параболический профиль:
	
	\begin{equation}
		u(r) = \frac{\Delta P}{4\eta l} (R^2 - r^2), \tag{3}
	\end{equation}
	
	где $l$ — длина трубы, $R$ — радиус. Интегрирование даёт объёмный расход:
	
	\begin{equation}
		Q = \int_0^R u(r) \cdot 2\pi r\, dr = \frac{\pi R^4 \Delta P}{8\eta l}. \tag{4}
	\end{equation}
	
	Это соотношение называют формулой Пуазейля. Заметим, что средняя скорость потока при пуазейлевском течении оказывается вдвое меньше максимальной:
	
	\begin{equation}
		\bar{u} \equiv \frac{Q}{\pi R^2} = \frac{u_{\text{max}}}{2}. \tag{5}
	\end{equation}
	
	\subsection{Длина установления течения}
	
	Пусть на вход трубы поступает течение, распределение скоростей которого не является пуазейлевским (например, равномерное). Ясно, что профиль течения не может установиться сразу, а реализуется лишь на некотором расстоянии $l_{\text{уст}}$ от начала трубы.
	
	Точный численный коэффициент аналитически установить затруднительно. Как показывает опыт, этот коэффициент можно с удовлетворительной точностью принять равным 0,2:
	
	\begin{equation}
		l_{\text{уст}} \approx 0,2R \cdot \text{Re}. \tag{6}
	\end{equation}
	
	Заметим, что если длина трубы мала по сравнению с $l_{\text{уст}}$, то работой сил трения в ней можно пренебречь и течение будет описываться не формулой Пуазейля, а уравнением Бернулли (при условии, что течение остаётся ламинарным).
	
	\subsection{Вязкость газов}
	
	Молекулы газа участвуют как в направленном движении со средней скоростью потока $u$, так и в хаотическом тепловом движении, характеризующемся средней тепловой скоростью $\bar{v} = \sqrt{\frac{8k_B T}{\pi m}}$ (здесь $m$ — масса молекулы). Молекулы могут свободно перемещаться между слоями и обмениваться друг с другом импульсами при столкновениях. Если в двух соседних слоях потока скорости различны, то такой обмен импульсом приводит к возникновению вязкости. Кинетическая теория даёт оценку:
	
	\begin{equation}
		\eta \sim \frac{1}{3} \rho \bar{v} \lambda, \tag{7}
	\end{equation}
	
	где $\lambda$ — средняя длина свободного пробега молекулы.
	
	\subsection{Турбулентность}
	
	При превышении некоторого критического числа Рейнольдса $\text{Re} > \text{Re}_{\text{кр}}$ течение Пуазейля становится неустойчивым. В потоке начинают рождаться вихри, которые затем сносятся вниз по трубе (при докритических числах Рейнольдса такие вихри быстро затухают за счёт вязкости). С дальнейшим увеличением Re количество вихрей возрастает и, взаимодействуя между собой, они порождают вихри всё меньшего размера, создавая таким образом сложную много масштабную картину течения.
	
	Примем, что флуктуации скорости в развитом турбулентном течении по порядку величины совпадают со средней скоростью потока: $\Delta u \sim \bar{u}$. При этом элементы жидкости практически равномерно перемешиваются по сечению трубы, так что в качестве «длины пробега» жидкой частицы можно взять радиус трубы:
	
	\begin{equation}
		\eta_{\text{турб}} \sim \rho \bar{u} R. \tag{8}
	\end{equation}
	
	Далее по аналогии с выводом формулы Пуазейля запишем баланс сил в потоке, откуда получим оценку для средней скорости течения:
	
	\begin{equation}
		\eta_{\text{турб}} \frac{\bar{u}}{R} \cdot 2\pi R l \sim \pi R^2 \Delta P \quad \to \quad \bar{u} \sim \frac{R^2 \Delta P}{\eta_{\text{турб}} l}. \tag{9}
	\end{equation}
	
	Подставляя сюда (8), находим скорость $\bar{u} \sim \sqrt{\frac{R \Delta P}{\rho l}}$ и, как следствие, расход:
	
	\begin{equation}
		Q = \pi R^2 \bar{u} \sim R^{5/2} \sqrt{\frac{\Delta P}{\rho l}}. \tag{10}
	\end{equation}
	
	\section{Экспериментальная установка}
	
	Поток воздуха под давлением, немного превышающим атмосферное, поступает через газовый счётчик в тонкие металлические трубки. Воздух нагнетается компрессором, интенсивность подачи регулируется краном К. Трубки снабжены съёмными заглушками на концах и рядом миллиметровых отверстий, к которым можно подключать микроманометр. В рабочем состоянии открыта заглушка на одной (рабочей трубке), микроманометр подключен к двум ее выводам, а все остальные отверстия плотно закрыты пробками. (Приложение 1)
	
	Перед входом в газовый счётчик установлен водяной U-образный манометр. Он служит для измерения давления газа на входе, а также предохраняет счётчик от выхода из строя. При превышении максимального избыточного давления на входе счётчика (~ 30 см вод. ст.) вода выплёскивается из трубки в защитный баллон Б, создавая шум и привлекая внимание экспериментатора.
	
	\textbf{Газовый счётчик.} В работе используется газовый счётчик барабанного типа, позволяющий измерять объём газа $\Delta V$, прошедший через систему. Работа счётчика основана на принципе вытеснения: на цилиндрической ёмкости укреплены лёгкие чаши, в которые поочерёдно поступает воздух из входной трубки расходомера. Когда чаша наполняется, она всплывает и её место занимает следующая и т.д. Вращение оси передаётся на счётно-суммирующее устройство. Для корректной работы счётчик должен быть заполнен водой и установлен горизонтально по уровню.(Приложение 2)
	
	\textbf{Микроманометр.} В работе используется жидкостный манометр с наклонной трубкой. Разность давлений на входах манометра измеряется по высоте подъёма рабочей жидкости (как правило, этиловый спирт). Регулировка наклона позволяет измерять давление в различных диапазонах. На крышке прибора установлен трехходовой кран, имеющий два рабочих положения — (0) и (+). В положении (0) производится установка мениска жидкости на ноль. В положении (+) производятся измерения.
	\section{Методика измерений}
	\section{Результаты измерений и обработка данных}
	
	\begin{table}[h]
		\centering
		\caption{Геометрические параметры трубок}
		\begin{tabular}{|c|c|c|}
			\hline
			& $d$, мм & $\sigma$, мм \\
			\hline
			Трубка 1 & 3,95 & 0,5 \\ 
			\hline
			Трубка 2 & 3,00 & 0,1 \\
			\hline
			Трубка 3 & 5,10 & 0,05 \\
			\hline
		\end{tabular}
	\end{table}
	
	Для перевода показаний микроманометра в паскали использовалось соотношение:
	
	\begin{equation}
		\Delta P = 9.8067 \cdot N \cdot K = 3.92 \cdot N, \tag{11}
	\end{equation}
	
	где $N$ — количество делений по шкале прибора, $K = 0.4$ — фактор наклона.
	
	Экспериментальные зависимости $Q(\Delta P)$ представлены на графиках (Приложение 3). В области ламинарного течения наблюдается линейный характер, что соответствует закону Пуазейля. Используя угловые коэффициенты $k$ линейных участков, динамическую вязкость вычисляли по формуле:
	
	\begin{equation}
		\eta = \frac{\pi R^4}{8 k l}, \tag{12}
	\end{equation}
	
	где $l$ — протяжённость рабочего участка, $R = d/2$ — радиус трубки.
	
	Полученные значения вязкости для каждой трубки:
	
	 Для трубки 1 ($d = 3.95$ мм, $R = 1.975$ мм, $l = 0.50$ м): $\eta_1 = (2.10 \pm 0.10) \cdot 10^{-5}$ Па·с.
	 
	 Для трубки 2 ($d = 3.00$ мм, $R = 1.500$ мм, $l = 0.30$ м): $\eta_2 = (1.97 \pm 0.09) \cdot 10^{-5}$ Па·с.
	 
	 Для трубки 3 ($d = 5.10$ мм, $R = 2.550$ мм, $l = 0.50$ м): $\eta_3 = (2.86 \pm 0.13) \cdot 10^{-5}$ Па·с.

	
	Средняя величина коэффициента вязкости воздуха:
	
	\begin{equation}
		\eta = (2.21 \pm 0.30) \cdot 10^{-5} \text{ Па·с}. \tag{13}
	\end{equation}
	
	Данный результат находится в удовлетворительном согласии с табличным справочным значением для воздуха при комнатной температуре ($\approx 1.8 \cdot 10^{-5}$ Па·с) с учётом погрешностей эксперимента и возможных вариаций условий (температура, влажность).
	
	Теоретические оценки параметров течения для трёх трубок:
	
	\begin{table}[h]
		\centering
		\caption{Теоретические параметры}
		\begin{tabular}{|c|c|c|c|c|c|}
			\hline
			$d$, мм & $\bar{u}$ & $l_{\text{уст}}$, м & $Q$, м$^3\cdot10^{-5}$ & $\Delta P$, Па & $Q$, л/мин \\
			\hline
			5.1 & 0.0163 & 0.5 & 6.5 & 139.2 & 3.91 \\ \hline
			3.95 & 0.0163 & 0.3 & 3.83 & 115.6 & 2.30 \\ 
			\hline
		\end{tabular}
	\end{table}
	
	В ходе измерений получены экспериментальные данные объёмного расхода $Q$ при различных перепадах давления $\Delta P$, $Q$ выражено в л/мин, $\Delta P$ — в делениях микроманометра.
	
	\begin{table}[h]
		\centering
		\caption{Экспериментальные значения $Q(\Delta P)$ для трёх трубок}
		\begin{tabular}{|c|c|c|c|c|c|}
			\hline
			$Q_1$ & $\Delta P_1$ & $Q_2$ & $\Delta P_2$ & $Q_3$ & $\Delta P_3$ \\
			\hline
			0.60 & 3 & 0.38 & 10 & 0.66 & 10 \\ \hline
			1.41 & 6 & 0.63 & 15 & 1.05 & 15 \\ \hline
			2.13 & 9 & 0.90 & 20 & 1.47 & 20 \\ \hline
			2.94 & 12 & 1.14 & 25 & 1.83 & 25 \\ \hline
			3.63 & 15 & 1.32 & 30 & 2.22 & 30 \\ \hline
			4.38 & 18 & 1.56 & 35 & 2.58 & 35 \\ \hline
			4.80 & 20 & 1.77 & 40 & 2.97 & 40 \\ \hline
			5.60 & 25 & 2.22 & 50 & 3.66 & 50 \\ \hline
			6.00 & 30 & 2.55 & 60 & 4.44 & 60 \\ \hline
			6.40 & 35 & 2.82 & 70 & 5.04 & 70 \\ \hline
			7.02 & 40 & 3.06 & 80 & 5.61 & 80 \\ \hline
			8.40 & 50 & 3.42 & 90 & 6.00 & 90 \\ \hline
			9.00 & 60 & 3.68 & 100 & 6.24 & 100 \\ \hline
			9.51 & 70 & 3.92 & 110 & 6.42 & 110 \\ \hline
			10.05 & 80 & 4.12 & 120 &  &  \\ \hline
			10.74 & 90 & 4.32 & 130 &  &  \\ \hline
			11.49 & 100 & 4.50 & 140 &  &  \\ \hline
			12.84 & 110 &  &  &  &  \\
			\hline
		\end{tabular}
	\end{table}
	
	Графические зависимости (Приложение 3) позволяют визуально определить момент перехода от ламинарного режима к турбулентному. Оценка критических перепадов даёт следующие значения: $\Delta P_{\text{кр1}} \approx 20$ дел, $\Delta P_{\text{кр2}} \approx 50$ дел, $\Delta P_{\text{кр3}} \approx 60$ дел. Наблюдается некоторое расхождение с теоретическими предсказаниями.
	

	С использованием формулы Пуазейля вязкость выражается как
	
	\begin{equation}
		\eta = \frac{\pi R^4}{8l \frac{Q}{\Delta P}} = \frac{\pi R^4}{8l k},
	\end{equation}
	
	где $k = Q/\Delta P$ — тангенс угла наклона линейного участка. Относительная погрешность определения вязкости:
	
	\begin{equation}
		\varepsilon_\eta = 4\varepsilon_R + \varepsilon_l + \varepsilon_k \approx 0.09 \quad \Rightarrow \quad \sigma_{\eta} = 0.17 \cdot 10^{-5} \ \text{Па}\cdot\text{с}.
	\end{equation}
	
	Численные значения угловых коэффициентов $k_i$ (в единицах СИ) составили:
	
	\begin{equation*}
		k_1 = 2.05 \cdot 10^{-6} \ \frac{\text{м}^3}{\text{Па}\cdot\text{с}}, \quad 
		k_2 = 3.75 \cdot 10^{-7} \ \frac{\text{м}^3}{\text{Па}\cdot\text{с}}, \quad 
		k_3 = 6.3 \cdot 10^{-7} \ \frac{\text{м}^3}{\text{Па}\cdot\text{с}}.
	\end{equation*}
	
	В таблице ниже обобщены результаты обработки.
	
	\begin{table}[!h]
		\centering
		\caption{Сводная таблица результатов}
		\begin{tabular}{|c|c|c|c|}
			\hline
			& Трубка 1 & Трубка 2 & Трубка 3 \\
			\hline
			$d$, мм & 5.1 & 3.95 & 3.0 \\ \hline
			$k\cdot10^6$, м$^3$/(Па·с) & 2.06 & 0.375 & 0.63 \\ \hline
			$\eta\cdot10^5$, Па·с & 1.6 & 1.8 & 1.9 \\ \hline
			$\text{Re}_{\text{кр}}$ & 1498 & 1047 & 1507 \\
			\hline
		\end{tabular}
	\end{table}
	\newpage
	Оценка погрешности критического числа Рейнольдса:
	
	\begin{equation}
		\varepsilon_{\text{Re}} = \varepsilon_{\rho} + \varepsilon_{Q_{\text{кр}}} + \varepsilon_{\eta} + \varepsilon_{d} \approx 0.16 \quad \Rightarrow \quad \sigma_{\text{Re}} = 240.
	\end{equation}
	
	Дополнительно исследовалась зависимость перепада давления $\Delta P$ от длины $l$ для фиксированных расходов. Полученные данные приведены в таблицах:
	
	\begin{table}[h]
		\centering
		\caption{Измерения $\Delta P(l)$ для $d_1$, $Q = 2.55$ л/мин}
		\begin{tabular}{|c|c|c|c|c|}
			\hline
			$l$, см & 10.7 & 30 & 40 & 50 \\
			\hline
			$\Delta P$, дел & 17 & 15 & 15 & 20 \\
			\hline
		\end{tabular}
	\end{table}
	
	\begin{table}[h]
		\centering
		\caption{Измерения $\Delta P(l)$ для $d_2$, $Q = 2.55$ л/мин}
		\begin{tabular}{|c|c|c|c|}
			\hline
			$l$, см & 11 & 20 & 30 \\
			\hline
			$\Delta P$, дел & 72 & 56 & 60 \\
			\hline
		\end{tabular}
	\end{table}
	
	\begin{table}[h]
		\centering
		\caption{Измерения $\Delta P(l)$ для $d_3$, $Q = 3.30$ л/мин}
		\begin{tabular}{|c|c|c|c|c|}
			\hline
			$l$, см & 10.9 & 30 & 40 & 50 \\
			\hline
			$\Delta P$, дел & 24 & 29 & 38 & 45 \\
			\hline
		\end{tabular}
	\end{table}
	
	Соответствующие графические зависимости $P(x)$ в приложениях 4-6.
	
	Из графиков получены значения $l_{уст1}=0.6 \pm 0.1 м$
	$l_{уст2}=0.3 \pm 0.1 м$
	$l_{уст3}=0.4 \pm 0.1 м$
	При одинаковых $\lambda=\Delta P / l$ ($\lambda_1=30 дел/м$, $\lambda_2=220 дел/м$) Построены графики логарифмированных значений турбулентного и ламинарного потоков(Приложения 7):
	
	Анализ полученных кривых немного завышен относительно теоретических ожиданий: в ламинарном режиме $Q \propto r^4$, в турбулентном $Q \propto r^{2.5}$. Отличия заключаются менее чем в 0,5 порядках (4.52 для ламинарного и 2.93 для турбулентного режимов)
	
	\section{Выводы и результаты}
	
	\begin{itemize}
		\item Экспериментально определённая динамическая вязкость воздуха, рассчитанная по формуле Пуазейля, составила $(1.77 \pm 0.17) \times 10^{-5}$ Па·с. Это значение хорошо согласуется с общепринятой величиной $1.8 \times 10^{-5}$ Па·с.
		\item Критическое число Рейнольдса, соответствующее переходу от ламинарного режима к турбулентному, найдено равным $\text{Re}_{\text{кр}} = 1350 \pm 270$, что по порядку величины соответствует ожидаемому значению $\approx 1000$.
		\item Подтверждена линейная связь $Q(\Delta P)$ в области ламинарного течения, что служит экспериментальным обоснованием закона Пуазейля.
		\item Для всех исследованных трубок получена линейная зависимость $P(x)$ при фиксированном расходе.
		\item Из анализа в логарифмическом масштабе установлены показатели степени при радиусе: $4.3 \pm 0.35$ для ламинарного потока и $2.8 \pm 0.35$ для турбулентного, что выше, чем теоретические значения на 20\%.
	\end{itemize}
	
	\section{Приложения}
	
	\begin{figure}[h] % Окружение для подписи и выравнивания
		\centering
		\includegraphics[width=0.5\textwidth]{Ust.png} % Имя файла без расширения
		\caption*{Приложение 1: Схема установки}
	\end{figure}

	\begin{figure}[h] % Окружение для подписи и выравнивания
		\centering
		\includegraphics[width=0.5\textwidth]{Gaz.png} % Имя файла без расширения
		\caption*{Приложение 2: Газосчетчик}
	\end{figure}
	
	\begin{figure}[h] % Окружение для подписи и выравнивания
		\centering
		\includegraphics[width=1\textwidth]{Figure_1.png} % Имя файла без расширения
		\caption*{Приложение 3}
	\end{figure}

	\begin{figure}[h] % Окружение для подписи и выравнивания
	\centering
	\includegraphics[width=0.5\textwidth]{Figure_2.png} % Имя файла без расширения
	\caption*{Приложение 4}
\end{figure}
	\begin{figure}[h] % Окружение для подписи и выравнивания
	\centering
	\includegraphics[width=0.5\textwidth]{Figure_3.png} % Имя файла без расширения
	\caption*{Приложение 5}
\end{figure}
	\begin{figure}[h] % Окружение для подписи и выравнивания
	\centering
	\includegraphics[width=0.5\textwidth]{Figure_4.png} % Имя файла без расширения
	\caption*{Приложение 6}
\end{figure}
	\begin{figure}[h] % Окружение для подписи и выравнивания
	\centering
	\includegraphics[width=1\textwidth]{Figure_5.png} % Имя файла без расширения
	\caption*{Приложение 7}
\end{figure}
\end{document}
