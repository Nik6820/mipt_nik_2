\documentclass[a4paper]{article}
\usepackage[warn]{mathtext}
\usepackage{graphicx} % Required for inserting images
\usepackage{wrapfig}
\usepackage{enumitem}
\usepackage[center]{caption}
\usepackage[english, russian]{babel}
\usepackage{amsmath}
\usepackage{geometry}
\setcounter{totalnumber}{5}

\geometry{verbose, a4paper, tmargin=2cm, bmargin=2cm, lmargin=2.0cm, rmargin=2.0cm}

\begin{document}
	
	\begin{titlepage}
		\centering
		{\scshape\Large МОСКОВСКИЙ ФИЗИКО-ТЕХНИЧЕСКИЙ ИНСТИТУТ \\
			(НАЦИОНАЛЬНЫЙ ИССЛЕДОВАТЕЛЬСКИЙ УНИВЕРСИТЕТ)\\ % название ВУЗа большим шрифтом
			Физтех-школа аэрокосмических технологий}
		
		\vspace{4cm} % отступ 4 см по вертикали
		{\LARGE Отчет о выполнении лабораторной работы 2.1.3}
		\vspace{1cm} % ещё 1 см
		
		{\huge\bf Определение $C_p/C_v$ по скорости звука в газе}
		
		\vspace{1cm} % ещё 1 см
		\vfill % заполнение по вертикали пустотой (LaTeX сам решит, сколько здесь отступить)
		
		\begin{flushright} % Этот текст будет с правого края страницы
			{\LARGE Губарев Никита, Б03-502}
		\end{flushright}
		\vfill 
		\today % Дата сборки документа
	\end{titlepage}
	\tableofcontents 
	\section{Аннотация}
	Цель работы: измерение частоты колебаний и длины волны при резонансе звуковых колебаний в газе, определение показателя адиабаты в приближении идеального газа.
	
	\section{Описание работы}
	\subsection{Теоретические сведения}
	Скорость распространения звуковой волны в газах зависит от показателя адиабаты $\gamma = C_p/C_v$:
	\begin{center} $c=\sqrt{\gamma\frac{RT}{\mu}}$ (1)\end{center} 
	где $R$ - газовая постоянная, $T$ - температура газа, $\mu$ - молярная масса. Преобразуя формулу:
	
	\begin{center}$\gamma = \frac{\mu}{RT}c^2$ (2)\end{center} 

	Полагая молярную массу известной, для определения показателя адиабаты достаточно измерить температуру газа и скорость распространения звука.
	
	Звуковая волна, распространяющаяся вдоль трубы с газом испытывает многократные отражения от торцов. Если длина трубы равна целому числу полуволн, то анализ выходного сигнала заметно упрощается:
	\begin{center}$L=n\frac{\lambda}{2}$ (3)\end{center} 
	
	где $L$ - длина трубы, $\lambda$ - длина волны звука, $n$ - любое целое число полуволн.
	Если это соотношение выполняется, то наступает резонанс. 
	
	Скорость звука $c$ связана с его частотой $f$ и длиной волны $\lambda$ соотношением:
	\begin{center}$c=\lambda f$ (4)\end{center} 
	
	При постоянной длине трубы можно изменять частоту звуковых колебаний. В этом случае следует плавно изменять частоту звукового сигнала, а следовательно и длину волны. Для последовательных резонансов получим:
	
	\begin{center}$L=\frac{\lambda_1}{2}n=\frac{\lambda_2}{2}(n+1)=...=\frac{\lambda_{k+1}}{2}(n+k)$ (5)\end{center} 
	
	Из 4 и 5 получим:
	\begin{center}$f_1=\frac{c}{\lambda_1}=\frac{c}{2L}n, \ f_2=\frac{c}{\lambda_2}=\frac{c}{2L}(n+1)=f_1+\frac{c}{2L}$\end{center} 
	\begin{center}$f_{k+1}=\frac{c}{\lambda_{k+1}}=\frac{c}{2L}(n+k)=f_1+\frac{c}{2L}k$ (6)\end{center} 
	
	Найдя коэффициент наклона графика зависимости частоты от номера резонанса, можно найти скорость звука для данных условий умножив полученный результат на удвоенную длину трубы.
	
	\subsection{Экспериментальная установка}
	
	Установка состоит из теплоизолированной трубы постоянной длины. Воздух внутри нагревается водой из термостата. Температура газа принимается равной температуре воды омывающей трубу. Звуковые колебания возбуждаются телефоном и улавливаются микрофоном, находящимися на разных торцах трубы. Сигнал на телефон подается с звукового генератора, микрофон передает сигнал на осциллограф (Приложение 1).
	\section{Методика измерений}
	\begin{enumerate}
		\item Включить генератор сигнала, осциллограф, термостат. Установить температуру 25$^{\circ}C$ 
		\item Измерить скорость звука в трубе. Увеличивая частоту генератора, получить ряд последовательных резонансных значений частоты.
		\item Полученные значения изобразить на графике частоты от номера резонанса. Провести прямую через полученные точки. Рассчитать скорость звука при данной температуре.
		\item Повторить измерения при различных температурах газа. 
		\item Вычислить значение $\gamma = C_p/C_v$.
	\end{enumerate}
	
	\section{Результаты измерений и обработка данных}
	Длина трубы $L=700 \pm 1 мм$ \\ Значения резонансных частот при различных температурах
	
	\begin{table}[h]
		\begin{tabular}{|c|c|c|c|c|c|}
			\hline
			$T \pm0,1 ^{\circ}C$ & $f_1 \pm 1, Гц$ & $f_2 \pm 1, Гц$ & $f_3 \pm 1, Гц$ & $f_4 \pm 1, Гц$ & $f_5 \pm 1, Гц$ \\ \hline
			25,4                 & 244             & 494             & 744             & 992             & 1238            \\ \hline
			35,0                 & 248             & 503             & 756             & 1007            & 1257            \\ \hline
			45,0                 & 251             & 512             & 768             & 1023            & 1277            \\ \hline
			55,0                 & 265             & 519             & 780             & 1038            & 1297            \\ \hline
		\end{tabular}
	\end{table}
	

	
	Значения угла наклона графика в зависимости от температуры (Приложения 2-5), скорость звука, показатель адиабаты (считаем, что газ внутри трубы - воздух с молярной массой 0.029 кг/моль):
	\begin{table}[h]
		\begin{tabular}{|l|l|l|l|l|}
			\hline
			$T, ^{\circ}C$ & 25,4 $\pm0,1$           & 35,0     $\pm0,1$         & 45,0      $\pm0,1$        & 55,0    $\pm0,1$          \\ \hline
			k, Гц                & $248,6 \pm 0,5$ & $252,2 \pm 0,6$ & $256,3 \pm 0,8$ & $258,3 \pm 0,6$ \\ \hline
			c, м/с               & $348,0 \pm 1,2$ & $253,1 \pm 1,3$ & $258,8 \pm 1,6$ & $361,6 \pm 1,3$ \\ \hline
			$\gamma$, о.е.             & $1,42 \pm 0,01$ & $1,41 \pm 0,02$ & $1,41 \pm 0,02$ & $1,39 \pm 0,02$ \\ \hline
		\end{tabular}
	\end{table}
	
	\section{Вывод}
	
	В ходе лабораторной работы были найдены скорости звука в зависимости от температуры, посчитаны для них показатели адиабаты. Значения показателей адиабаты лежат в пределах табличного значения 1,4. Скорости звука близки к табличными (346 м/с, при  $25^{\circ}C$; 352 м/с, при  $35^{\circ}C$; 358 м/с, при  $45^{\circ}C$; 364 м/с, при  $55^{\circ}C$). Погрешность составляет 0,2-0,7\%.
	
	\newpage
	\section{Приложения}
	\begin{figure}[h] % Окружение для подписи и выравнивания
		\centering
		\includegraphics[width=0.8\textwidth]{scheme.png} % Имя файла без расширения
		\caption*{Приложение 1: Схема установки}
	\end{figure}
	\begin{figure}[h] % Окружение для подписи и выравнивания
		\centering
		\includegraphics[width=1\textwidth]{25.png} % Имя файла без расширения
		\caption*{Приложение 2: График для $25^{\circ}C$}
	\end{figure}
	\begin{figure}[h] % Окружение для подписи и выравнивания
	\centering
	\includegraphics[width=1\textwidth]{35.png} % Имя файла без расширения
	\caption*{Приложение 3: График для $35^{\circ}C$}
	\end{figure}
	\begin{figure}[h] % Окружение для подписи и выравнивания
	\centering
	\includegraphics[width=1\textwidth]{45.png} % Имя файла без расширения
	\caption*{Приложение 4: График для $45^{\circ}C$}
	\end{figure}
	\begin{figure}[h] % Окружение для подписи и выравнивания
	\centering
	\includegraphics[width=1\textwidth]{55.png} % Имя файла без расширения
	\caption*{Приложение 5: График для $55^{\circ}C$}
	\end{figure}
\end{document}

